\cleardoublepage
\chapternonum{摘要}

仿人机器人是机器人领域的研究热点之一。仿人机器人存在机械结构复杂,动力学
方程复杂,稳定控制要求较高等特点,其运动的稳定性至今无法达到人们的期望,其中
走上台阶更是机器人运动控制的重要挑战。本文针对仿人机器人开展基于优化的运动控制研究,设计仿人机器人状态估计器,基于二次规划的运动控制器
,以及基于轨迹优化的质心轨迹规划器,并在仿人机器人WuKong-IV平台上的开展实验验证。本文的主要研究内容有:

1) 设计了仿人机器人状态估计器。首先对仿人机器人每条腿触地的概率进行了估计,
为步行状态机提供了切换信号;其次对仿人机器人质心在世界系下的位置、速度进行了估计
为后续的运动控制和规划建立了基础。

2) 设计了基于二次规划的仿人机器人运动控制器。将仿人机器人每条腿的运动分为支撑相和摆动相,
摆动相采用位置控制跟随根据质心速度计算出的落脚点位置;支撑相将上层规划器规划出的运动行为
组合成一个二次规划问题,求解出关节的期望力矩作为电机的前馈输入,状态反馈得到的期望力矩作为
电机的反馈输入。

3) 设计了基于轨迹优化的仿人机器人运动规划器。将变高度双摆模型作为规划的动力学模型,将力矩和腿长约束、
滑动摩擦约束、转动摩擦约束作为优化问题的约束,并设计一系列的目标函数以及正则项,最终构建出一个轨迹优化问题,
利用非线性规划求解器求解,最终得到机器人质心的期望轨迹。

4) 在WuKong-IV仿人机器人实验平台上开展实验,验证上述控制算法。成功验证
状态估计器的准确性,以及运动控制器,运动规划器的有效性。

关键词:WuKong-IV仿人机器人;二次规划;轨迹优化;运动控制

\cleardoublepage
\chapternonum{Abstract}

Biped robot is one of the research hotspots in the field of robotics. Biped robots have the characteristics of complex mechanical structures, 
complex dynamic equations, and high requirements for stability control. The stability of their motion has so far failed to meet people's expectations. 
Walking up steps is an important challenge for robot motion control. This thesis conducts optimization-based motion control research for biped robots, 
designs a biped robot state estimator, a motion controller based on quadratic programming, and a center-of-mass trajectory planner based on trajectory optimization. 
carry out experimental verification. The main research contents of this thesis are:

1) A biped robot state estimator is designed. First, the probability of each leg of the biped robot touching the ground is estimated, 
which provides a switching signal for the walking state machine; secondly, the position and velocity of the center of mass of the biped robot in the world system are estimated, 
which provides a basis for subsequent motion control and planning. The foundation is established.

2) A biped robot motion controller based on quadratic programming is designed. The motion of each leg of the biped robot is divided into a support phase and a swing phase. 
The swing phase uses position control to follow the foothold position calculated according to the velocity of the center of mass; the support phase combines the motion behavior 
planned by the upper planner into a quadratic plan The problem is to solve the expected torque of the joint as the feedforward input of the motor, and the expected torque 
obtained by the state feedback as the feedback input of the motor.

3) A biped robot motion planner based on trajectory optimization is designed. The variable-height double pendulum model is used as the dynamic model for planning, 
the torque and leg length constraints, sliding friction constraints, and rotational friction constraints are used as constraints for the optimization problem, 
and a series of objective functions and regular terms are designed to finally construct a trajectory optimization The problem is solved using a nonlinear programming 
solver, and finally the expected trajectory of the robot's center of mass is obtained.

4) Experiments were carried out on the WuKong-IV biped robot experimental platform to verify the above control algorithm. 
Successfully verified the accuracy of the state estimator, and the effectiveness of the motion controller, motion planner.

Key words: WuKong-IV biped robot; Quadratic programming; Trajectory Optimization; Locomotion control