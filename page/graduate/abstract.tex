\cleardoublepage
\chapternonum{摘要}
双足机器人是机器人领域的研究热点之一。双足机器人存在机械结构复杂,动力学
方程复杂,稳定控制要求较高等特点,其运动的稳定性至今无法达到人们的期望,其中
双足机器人的跑跳步态更是机器人运动控制的重要挑战。本文针对双足机器人的跑跳运
动开展步态规划和平衡控制研究,设计双足机器人状态机,单腿支撑过程中的身体姿态
平衡与质心高度平衡控制,摆动腿的运动规划,以及机器人在低速状态下的方向控制。
最终在双足机器人“悟空-II”平台上的开展跑跳实验验证。本文的主要研究内容有:

1) 构造了双足机器人跑跳步态的状态机,设计姿态平衡控制器和高度控制器。将
双足机器人前进过程分为单腿支撑相与双腿腾空相,机器人处于单腿支撑相时,
设计控制器保持身体的姿态;利用对地面反作用力的控制维持身体质心高度,
为机器人能够采用跑跳步态前进提供前提。

2) 设计了双足机器人快速运动的速度控制器。描述结合运动学与陀螺仪数据,采
用卡尔曼滤波器估计质心速度的方法;重点介绍摆动腿的轨迹规划与落脚点控
制,根据LIP 模型控制,利用机器人摆动腿的落脚点实现对机器人的速度控制,
同时规划摆动腿摆动曲线以及添加摆动腿摆动过程中的重力补偿。

3) 设计了双足机器人的转向运动控制。针对机器人前进过程中会出现方向偏差的
问题,采用对支撑腿髋关节偏航关节的运动规划实现机器人低速运动下的方向
控制;机器人高速前进时,采用利用腰关节转动平衡机器人支撑腿与摆动腿运
动产生的转动力矩;同时将双足机器人的偏转力矩加入机器人的姿态控制器中,
期望实现双足机器人的方向上的控制。

4) 在“悟空-II”双足机器人实验平台上开展实验,验证上述控制算法。成功验证
身体姿态控制器与质心高度控制器的有效性;利用卡尔曼滤波器获得更加准确
的质心速度,利用落脚点控制实现“悟空-II”机器人前进速度的控制;改进机
器人前进过程中的方向控制。

关键词:“悟空-II”双足机器人;姿态控制;落脚点控制;yaw 方向控制

\cleardoublepage
\chapternonum{Abstract}
Biped robot is one of the research hotspots in the field of robotics. The biped robot has the
characteristics of complex mechanical structure, complex dynamic equation and high stability
control requirements. The stability of its motion cannot meet people's expectations so far, and
the running and jumping gait of the biped robot is an important challenge for robot motion
control. In this paper, gait planning and balance control research are carried out for the running
and jumping motion of biped robot. The state machine of biped robot is designed, the body
posture balance and the height balance of mass center in the process of single leg support, the
motion planning of swinging leg, and the direction control of robot at low speed. Finally, the
running and jumping experiments are carried out on the platform of the biped robot WUKONGII.
The main contents of this thesis are as follows:

1) A state machine of running gait of biped robot is constructed. The attitude balance
controller and altitude controller are designed. The forward process of biped robot is
divided into single leg support phase and double leg flight phase. When the robot is
in single leg support phase, the controller is designed to maintain the posture of robot
body, and the control of the ground reaction force is used to maintain the height of
the body mass center. It provides the premise for the robot to move forward with
running gait.

2) The speed controller of biped robot is designed. Based on the kinematic and gyroscopic
data, Kalman filter is used to estimate the velocity of the center of mass. The
trajectory planning and the leg placement control of the swinging leg are described.
According to the LIP model control, the leg placement of the swinging leg is used to
realize the velocity control of the robot, and the swinging curve of the swinging leg
is planned and the gravity compensation during the swinging process is added.

3) In order to solve the problem of direction deviation in the process of robot moving
forward, the motion planning of the yaw joint of the support leg hip joint is used to
realize the direction control of the robot at low speed. When the robot is moving at
high speed, the rotation moment generated by the movement of the support leg and
the swing leg is balanced by the rotation of the waist joint. The deflection moment of
the biped robot is added to the attitude controller of the robot in order to achieve the
direction control of the biped robot.

4) Experiments are carried out on the WUKONG-II biped robot to verify the control
algorithm. The validity of the body attitude controller and the center of mass height
controller is verified successfully; the more accurate center of mass speed is obtained
by using Kalman filter, and the forward speed of WUKONG-II robot is controlled by
using the leg placement control; the direction control during the robot's forward process
is improved.

Key words: WUKONG-II biped robot; Attitude control; Leg placement control; Gravity
compensation; Yaw direction control