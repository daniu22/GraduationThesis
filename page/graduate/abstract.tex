\cleardoublepage
\chapternonum{摘要}

仿人机器人及其相关研究在机器人领域内方兴未艾。仿人机器人存在结构复杂,自由度高,非线性强,本质不稳定等显著特性,
想要实现其在各种环境下的稳定运动至今依然是一项重大挑战。本文针对仿人机器人轨迹跟踪和质心轨迹规划问题开展基于优化的运动控制研究,设计仿人机器人状态估计器、基于二次规划的运动控制器
、以及基于轨迹优化的质心轨迹规划器,并在仿人机器人“悟空-IV”平台开展实验验证。本文的主要研究工作和成果如下:

1) 设计了仿人机器人状态估计器。首先介绍了仿人机器人实验平台的相关硬件参数,利用机器人上已有的传感器硬件,
先对仿人机器人每条腿触地的概率进行了估计,为步行状态机提供了切换信号;其次对仿人机器人浮动基在世界系下的位置、速度进行了估计,
为后续的运动控制和轨迹规划建立了基础。

2) 设计了基于二次规划的仿人机器人步行控制器。定义了机器人步行运动的有限状态机,将仿人机器人行走时每条腿的运动分为支撑相和摆动相,
摆动相采用位置控制跟随根据质心速度计算出的落脚点位置;支撑相将上层规划器规划出的运动行为
组合成二次规划问题,利用qpOASES求解库求解,将求解出的关节的期望力矩作为关节的前馈输入,将状态反馈得到的期望力矩作为
关节的反馈输入,最终得到支撑腿各关节的实际期望力矩值。

3) 设计了基于轨迹优化的仿人机器人上台阶轨迹规划器。针对仿人机器人上台阶的稳定通行需求,将变高度双摆模型的动力学方程作为质心轨迹规划的动力学约束,以及机器人腿长约束、
关节最大力矩约束、滑动摩擦约束、转动摩擦约束作为优化问题的约束条件,并设计一系列的目标函数以及正则项,最终构建出一个轨迹优化问题,
利用Ipopt非线性优化求解器求解,最终得到登上台阶过程中机器人质心的期望轨迹。

4) 在“悟空-IV”仿人机器人实验平台上开展实验。实现了仿人机器人可靠的触地估计和浮动基位置速度估计;机器人最大稳定行走速度超过6km/h,并能实现误差在几厘米以内的位置跟踪;可以登上高度为10cm
的三级连续台阶,验证上述算法的有效性。

$\textbf{关键词}$:仿人机器人;状态估计;二次规划;轨迹优化;运动控制;

\cleardoublepage
\chapternonum{Abstract}

Humanoid robots and their related research are in the ascendant in the field of robotics. Humanoid robots have significant characteristics such as complex structure, 
high degree of freedom, strong nonlinearity, and inherent instability. It is still a major challenge to achieve stable movement in various environments. In this thesis, 
the optimization-based motion control research is carried out for the trajectory tracking and centroid trajectory planning problems of humanoid robots, 
and the state estimator for humanoid robots, the motion controller based on quadratic programming, and the centroid trajectory planner based on 
trajectory optimization are designed. The human robot "Wukong-IV" platform carried out experimental verification. The main research work and achievements of this thesis 
are as follows:

1) A state estimator for a humanoid robot is designed. Firstly, the relevant hardware parameters of the humanoid robot experiment platform are introduced. 
Using the existing sensor hardware on the robot, the probability of each leg of the humanoid robot touching the ground is estimated, 
which provides a switching signal for the walking state machine; The position and velocity of the human-robot floating base in the world system are estimated, 
which establishes the foundation for the subsequent motion control and trajectory planning.

2) A walking controller for humanoid robot based on quadratic programming is designed. Firstly, the finite state machine of the robot walking motion is defined, 
and the motion of each leg of the humanoid robot is divided into a support phase and a swing phase. The motion behavior planned by the planner is combined into a 
quadratic programming, and the qpOASES software library is used to solve the problem. The expected torque of the joint is used as the feed-forward input of the joint, 
and the expected torque obtained by the state feedback is used as the feedback input of the joint. Finally, the desired torque values are sent to each joint of the supporting leg.

3) A trajectory planner based on trajectory optimization for a humanoid robot going up stairs is designed. Aiming at the stable walking requirements of the humanoid robot 
on the stairs, the dynamic equation of the variable-height double pendulum model is used as the dynamic constraint of the center-of-mass trajectory planning, and the robot 
leg length constraint, joint maximum moment constraint, sliding friction constraint, and rotational friction constraint are optimized The constraint conditions of the problem, 
and a series of objective functions and regular terms are designed, and finally a trajectory optimization problem is constructed, which is solved by using the Ipopt nonlinear 
optimization solver, and finally the expected trajectory of the robot's center of mass in the process of climbing the steps is obtained.

4) Experiments were carried out on the "Wukong-IV" humanoid robot experimental platform. the reliable ground contact estimation and floating base position and velocity 
estimation of the humanoid robot were realized; the maximum stable walking speed of the robot exceeded 6km/h, and the error of position tracking was within a few centimeters; 
the robot can climb three consecutive steps with a height of 10cm each, which verifies the effectiveness of the above algorithm.
$\textbf{Keywords}$: Humanoid robot; State estimation; Quadratic programming; Trajectory optimization; Locomotion control