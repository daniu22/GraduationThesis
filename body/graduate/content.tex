\chapter{绪论}

% 本模板根据浙江大学研究生院编写的《浙江大学研究生学位论文编写规则》~\cite{zjugradthesisrules},
% 在原有的 zjuthesis 模板~\cite{zjuthesis}基础上开发而来。

% 本模板的本科生版本\cite{zjuthesisrules}得到了浙江大学本科生院老师的支持与审核,
% 已经在本科生院网上公示。
% 但当前的研究生版本并未经过研究生院老师的审核,
% 同学们使用时要注意对照模板与要求,
% 切不可盲目使用。

% 作者本人并未编写过浙江大学研究生毕业论文,
% 所以不清楚具体要求。
% 如果有热心同学愿意帮忙,
% 可以替我联系相关老师,我会配合审核并修改代码。

% \section{Overleaf 使用注意事项}

% 如果你在Overleaf上编译本模板,请注意如下事项:

% \begin{itemize}
%     \item 删除根目录的 ``.latexmkrc'' 文件,否则编译失败且不报任何错误
%     \item 字体有版权所以本模板不能附带字体,请务必手动上传字体文件,并在各个专业模板下手动指定字体。
%         具体方法参照 GitHub 主页的说明。
%     \item 当前的Overleaf默认使用TexLive 2017进行编译,但一些伪粗体复制乱码的问题需要TexLive 2019版本来解决。
%         所以各位同学可以在Overleaf上编写论文时务必切换到TexLive 2019或更新版本来编译,以免产生查重相关问题。
%         具体说明参照 GitHub 主页。
% \end{itemize}


% \section{节标题}

% 我们可以用includegraphics来插入现有的jpg等格式的图片,
% 如\autoref{fig:zju-logo}所示。

% \begin{figure}[htbp]
%     \centering
%     \includegraphics[width=.3\linewidth]{logo/zju}
%     \caption{\label{fig:zju-logo}浙江大学LOGO}
% \end{figure}


% \subsection{小节标题}


% \par 如\autoref{tab:sample}所示,这是一张自动调节列宽的表格。

% \begin{table}[htbp]
%     \caption{\label{tab:sample}自动调节列宽的表格}
%     \begin{tabularx}{\linewidth}{c|X<{\centering}}
%         \hline
%         第一列 & 第二列 \\ \hline
%         xxx & xxx \\ \hline
%         xxx & xxx \\ \hline
%         xxx & xxx \\ \hline
%     \end{tabularx}
% \end{table}


% \par 如\autoref{equ:sample},这是一个公式

% \begin{equation}
%     \label{equ:sample}
%     A=\overbrace{(a+b+c)+\underbrace{i(d+e+f)}_{\text{虚数}}}^{\text{复数}}
% \end{equation}

\section{腿足运动控制最新进展}
腿的作用在于,机器人能用腿上的末端执行器来推“地”从而为身体提供主动避震,
使得机器人身体的运动要比地表的轮廓更平滑。我们把所有用来推“地”的末端执行器统一叫做“足”,
不管它们的形式和功能具体是什么。腿可以暂时离开地面来迈步,因此可以通过不连续的地形,
在轮子无法到达的地方运动。无论腿的数量、运动学构型和自由度数,
足的类型(平面足,可变形足,点足,轮足)如何变化,根本的原理是一样的。
因此,最近几年原本在平面足人形机器人上发展的方法也快速适配了四足机器人以及其他形式足的腿足机器人。

控制的目标是找到足够多的可以在足和环境之间建立接触的位置,并且执行对应的动力学可行的轨迹。
为了处理腿足运动控制的关键难题——交替的触地和对应的接触力约束,
目前几乎推广到所有形式的腿足机器人的通用方法大量使用了数值轨迹优化及其在线实现——MPC。

除了将这种通用的方法适配到不同数量的腿和不同类型的足,
近几年在腿足运动控制领域的主要进展是如何用“提纯”的模型和“提纯”的数值模型来处理轨迹优化问题。
这使得从在多为平地的双足行走到更多样的机器人形态
,更通用的不平整地形的多接触运动的转换成为可能。
除了将这种通用的方法适配到不同数量的腿和不同类型的足,
近几年在腿足运动控制领域的主要进展是如何用“提纯”的模型和“提纯”的数值模型来处理
轨迹优化问题。这使得从在多为平地的双足行走到更多样的机器人形态
,更通用的不平整地形的多接触运动的转换成为可能。

\subsection{腿足运动控制的典型方法}

The dynamics of legged locomotion: 牛顿方程说明和环境接触产生的外力$f_i$减去重力$mg$决定了机器人质心$c$变化的方向:
\begin{equation}
    \label{equ:newton}
    m(\ddot{c}-g)=\sum_i f_i
\end{equation}
其中$m$是机器人的总质量。同时对应的欧拉方程表明接触点$s_i$相对质心$c$的位置是控制机器人身体相对质心的角动量$L$的关键:
\begin{equation}
    \label{equ:euler}
    \dot{L}=\sum_i\left(s_i-c\right) \times f_i
\end{equation}

Artificial synergy synthesis 方法认为,应该更加关注和地面接触力直接绑定的质心动力学
\autoref{equ:newton}和\autoref{equ:euler},
而不是和关节力矩更直接绑定的机器人的精准关节姿态\cite{vukobratovic1972contribution},如\autoref{fig:centroid}所示。
\begin{figure}[htbp]
    \centering
    \includegraphics[width=.5\linewidth]{图1.png}
    \caption{\label{fig:centroid}质心动力学及其与全身动力学的联系释图(图源~\cite{carpentier2016center})}
\end{figure}
这种方法在现在是机器人运动控制的核心,但在需要紧密协调机器人平衡和姿态的情况下,基于完整多刚体拉格朗日动力学的方法才是首选,
因为它天然包含了质心动力学\cite{orin2013centroidal}。

腿足运控的难题之一是接触力通常是单边的:机器人只能推接触表面,而不能拉。因此接触力$f_i$只能朝向特定的方向,由摩擦锥来约束。
另外,每个接触都是二元的:要么有接触和接触力,要么没有接触也没有接触力。当腿和某个面碰撞,也会产生冲击力。
触地转换和冲击是影响连续动力学\autoref{equ:newton}和\autoref{equ:euler}的离散事件,从这个角度讲腿足机器人的动力学是混合的,
这是难题之二。

理论上,从机器人可以执行合适的接触力来避免摔倒的状态中,我们可以定义可行状态(viable states)。周期运动(周期踏步?)和平衡点
显然是可行的,并且如果机器人能从给定状态在几步内达到这样一个循环或者是平衡点,那这样的状态也是可行的\cite{wieber2002stability}。
这就是目前许多现有腿足运控方法的关键\cite{wieber2016modeling}——可捕获性分析\cite{pratt2006velocity}的要义。

Control architecture: 接触点$s_i$的序列通常要考虑机器人所在环境及其目标点、运动学和静平衡约束\cite{escande2013planning},
用随机采样点的方法在时空域上预先规划好,这个过程称作落脚点规划。预规划好的落脚点在后续执行的过程中可以根据实际的地形
和外界扰动来调整cite{feng2016robust}。对应的质心运动和角动量可以通过在线MPC的框架来得到\cite{wieber2006trajectory},
该框架在考虑质心动力学\autoref{equ:newton}和\autoref{equ:euler}的同时需要保证机器人的状态始终是可捕获的\cite{wieber2016modeling}。

当机器人在平地上运动,所有接触点$s_i$都是共面的,接触力$f_i$通常会退化到他们的零力矩点(ZMP),而ZMP在接触点组成的凸包内[6,16]
(通过单边接触约束的多面体投影,非平面的接触点也有类似的性质[17])。在这种情况下,预先定义好质心在竖直方向上的运动可以得到高效的线性模型。
最常见的是线性倒立摆模型(LIP),它假设质心在平面内运动并且角动量为0[18]。考虑非零的角动量并不影响模型的线性,但很少有人这样做[19],
可能是在平地上行走时这样做的好处并不多[20]。

通常用逆运动学[21],或者笛卡尔空间下的全身动力学的反馈线性化[22],来让机器人的整个身体去执行规划出的质心运动,角动量和触地位置,
用二次规划(Quadratic Programs)来考虑瞬时的运动学、动力学、力和力矩约束[22,23,24]。分层QP可以在控制目标函数中考虑不同的优先级[25],
但存在奇异的问题[26]。逆运动学和反馈线性化对模型误差尤其是位置误差,以及对机器人平衡至关重要的地面接触的稳定性和刚度非常敏感。
基于欠驱动(passivity)的方法更不依赖于准确的接触模型[27,28,29],是上述方法很有前景的一种替代。
\begin{figure}[htbp]
    \centering
    \includegraphics[width=1.0\linewidth]{图2.png}
    \caption{\label{fig:typical_control}典型的控制框架由主要由三阶段组成:接触规划,质心动力学级规划和全身控制
                (图源~\cite{carpentier2016center})}
\end{figure}
\autoref{fig:typical_control}所示腿足运控的通用方法,原本是为双足行走开发的并且在Kawada的HRP-2[13],本田的Asimo[30],Aldebaran的Nao[31],
波士顿动力的Atlas[32]等等一系列的机器人上得到了成功的验证。最近这种方法也被适配到了诸如ANYbotics的ANYmal[33],IIT的HyQ[34],
MIT的Cheetah[35]等四足机器人以及Aldebaran的pepper[36],ETHZ的ANYmal[37]和Ascento[38]等轮足机器人上。

\subsection{落脚点规划的最近进展}
落脚点规划是运动规划的一个特例,这个问题通常是PSPACE难的,因此是NP难的[39]。在腿足运动中,接触状态的离散特性是组合爆炸的一个特殊来源,
增加了运动规划问题整体的非凸性。为了解决这种讨厌的计算复杂性,已经有人提出利用现成先进的求解器在几秒内找到全局最优的接触序列的
混合整数公式(mixed-integer formulations),同时还考虑了在不平整地形的运动学甚至动力学可行性约束[40,41,42]。

另外一种方法是用机器人足能到达的范围来提供接触序列存在的充分必要条件。这样机器人的运动可以不用提前设定好接触序列来实现,
这大大降低了运动规划问题的复杂度,使机器人可以在几分之一秒内对变化的环境作出快速重规划。具体的接触序列可以在下一个阶段用各种各样的
启发式函数来得到,比如在考虑不平整地形下运动学甚至动力学可行性约束的同时最大化准静态的鲁棒平衡[43,44,45,15]。这种多阶段的方法可以很高效,
但是对启发函数的依赖可能会导致失败。

\subsection{简化模型的近期进展}
运动发散量(divergent component of motion, DCM):线性倒立摆(LIP)模型的一个好处是它足够简单,可以进行彻底的数学分析。值得注意的是,
LIP动力学的发散部分可以用一个单独的量表示——运动发散量(DCM),而不必用质心的完整状态(位置和速度)。DCM的线性反馈控制是双足平地运动的
一种高效平衡控制策略[20],简单到可以在基于位置控制[46]、力矩控制[47]等各种各种机器人平台上验证。

DCM遵循接触力的一阶动力学,比质心的二阶动力学(1)要简单,这也帮助简化了腿足控制的其他方面。最近主要有三个进展:第一,DCM可以在
不改变系统动力学的前提下,去掉质心的平面约束来生成三维的行走轨迹[48];第二,通过对DCM的解析轨迹做一些变量替换,
可以在轨迹优化中包含在线的step-timing adaption的同时保证我们要解的问题是一个小的凸问题[49]。第三,在证明MPC问题的稳定性和可行性时,
DCM可以被用作一个可行(viability)条件。

Centroidal dynamics: 在例如不平整地形、狭窄空间等有多点接触的场景中运动时,将接触力等效到ZMP是很局限的。基于完整质心动力学
\autoref{equ:newton}和\autoref{equ:euler}的方法是在更广泛的接触场景下计算可行的质心轨迹、角动量和对应接触力的一种解决方案。
接触序列甚至可以同时优化求解[51]。

利用轨迹优化问题的稀疏性和凸性来求解的方法已经有了一定的发展[41,44,45]。另外,四足机器人的惯量通常用单刚体近似得到[34,35,51],
并且为了让最终的非线性优化问题可以在期望的运动行为附近快速收敛,可以根据实际运动设计启发函数来正则化优化问题[53]。加上这些改进之后,
基于完整质心动力学的方法可以在线运行,计算时间在10ms内[54]。

质心动力学并不考虑运动学或者关节力矩约束。为了部分弥补这些缺陷,一些方法提出将接触力和力矩(wrench)锥扩展来包含力矩限制[55],
还有一些基于学习的方法将更复杂的全身约束表示进质心模型里面[56]。

Mixed models: 另外一个在简化模型跟完整动力学模型之间的折中方案是,将质心动力学\autoref{equ:newton}和\autoref{equ:euler}和
全身运动学模型结合,这样可以在保持适度的计算复杂度的同时得到符合动力学的全身运动[57,58]。用这种方法可以在70ms内在线计算和更新
局部最优的反馈控制[59]。也可以用交替下降的方法(alternating descent method)将质心动力学与全身动力学模型结合,来考虑质心动力学以外的部分[60]。
一种更早的方法将机器人的瞬间全身动力学与LIP模型在一个滚动时域内结合到一个统一的QP问题中,来获得机器人运动的长时动力学[61]。

\subsection{全身模型的最新进展}
Trajectory optimization: 利用collocation[62,63]或multipe-shooting[64]等轨迹优化的一些技术,我们可以考虑所有的运动学和动力学约束,
以及最小化能量消耗等目标函数,用机器人的完整非线性动力学模型来协调肢体的运动(其中collocation和multipe-shooting在求解速度、准确性、
稳定性和精度上互有优势[65])。使用松弛的接触模型甚至能同时优化求解接触序列[66,67]。由于高效的开源数值求解器[68,69,70]利用了轨迹优化问题的
稀疏性和动力学模型的精确导数[71,72,73],在线计算全身运动可以在7ms内完成[70,74,75]。

Reinforcement learning:强化学习技术的成熟使得离线用机器人的完整动力学模型,从传感器原始数据和机器人姿态估计中直接计算出
最优的反馈控制策略成为可能[76,77]。这种方法在更难以建模的室内和室外环境中比更传统的方法表现要好。这种方法有效的关键点在于:
(1)能产生大量行为样本的高效仿真器[78];(2)能指引反馈控制策略的优化求解从简单场景到更复杂场景的“上课”策略;
(3)在电机模型里用实际的硬件测量数据训练得到的神经网络表示难以建模的不确定性。

\subsection{小结}
腿足运控就是在足和环境之间建立一系列的接触,并且在不同接触之间执行对应的符合机器人运动学和动力学的动作。数值轨迹优化在这其中扮演了关键角色,
处理交替的接触以及对应接触力的约束。不同复杂度的模型,从最简单的LIP模型,到完整质心动力学(1)-(2),再到完整的多刚体动力学模型,
可以单独或者结合起来使用,处理运动的不同方面(层次)。正是这些模型的改进,以及高级的数值方法的改进,无论是现成的还是特殊为腿足运控设计的,
使得近年来从平地到更普遍的室内室外环境的转变成为可能。首个用强化学习成功实现真实复杂环境下的四足运动的例子也是用同样的方法做到的。

离线和在线数值优化在这个过程中扮演的核心角色,使得整个腿足运控研究领域大力发展新的优化方法和高效的数值软件。值得一提的有,安全处理不同优先级、
多目标函数的Lexicographic规划,利用非线性轨迹优化问题稀疏性的微分动态规划(DDP),甚至还有将这两种方法相结合的[79]。
这其中涉及到包含动力学模型[80,81,82],仿真[83],状态估计[84],控制器设计[85]的灵活高效开源软件的发展。

开源软件,以及包含3D打印零件,简化的电路跟执行器的开源硬件[86,87],可以促进完善、鲁棒、经过大量测试的解决方案的复现、传播和平民化,
同时有利于快速迭代和创新,并且使得这些现有实现方法可以扩展和改进(表1列出了一部分)。

腿足运控控制的现有文献大多数都只用了状态反馈,但有意思的是由于质心是虚拟点,并且和环境的接触是很难检测和测量的,所以质心的位置和速度是
不能直接测量得到的。因此状态估计是另一个关键的部分,但这部分被探索的少得惊人[84,2,88,89,90,91,92]。对状态估计和反馈控制的依赖使得
现有的腿足机器人极度依赖昂贵、精细的硬件带来的准确信息。基于传感器的控制可能是另一种方案[93],很大程度上还尚待探索。
鲁棒控制也是一种可能的方案[94,95,96,97],同样尚待研究。腿足机器人的鲁棒控制分析得出的最新结论是,正如在人形[96]和四足机器人[59]上
已经验证过的那样,在低的惊人的频率下就可以得到完美稳定的反馈控制。

基于上文提到的开源软件和硬件的控制方法可以促进更便宜,更鲁棒,功能更多的机器人的发展。这也是腿足运动在室内外各种环境得到验证,
接近真实工业场景[98,99]后将会迈出的很合理的一步。可变形的元素也会推动这一目标[100]。


\chapter{双足机器人状态估计}
\section{质心状态估计}
包括人形机器人在内的腿足机器人有驱动关节,但并没有和环境刚性固定。因此驱动关节的角度并不能决定机器人相对环境的位姿。
腿足机器人的构型还需要确定六个自由度,通常是一个特殊的连杆相对某个惯性系的位置和姿态。这个连杆就是所谓的浮动基。
器人每个连杆的位置可以用浮动基的位姿和编码器测量得到的关节角度重建出来。

由于浮动基的状态并不是直接可测的,所以快速可靠的状态估计是在腿足机器人上实现状态反馈控制非常重要的一环。如果能知道触地点的精确位置,
那么关节的角度就足够重建出浮动基的位姿。但实际上接触很有可能会发生打滑,或者是踩到边缘脚板翻转,而且环境可能是不规则的,
机器人的几何模型中也可能有不确定性(比如柔性零件引起的变形)。因此关节角度信息是不足以实现可靠的状态估计的。

力/力矩传感器和惯性测量单元(IMU)可以帮忙解决这个问题。力测量可以用来检测触地,打滑,以及脚板翻转。这些测量也可以用来估计柔性零件的形变,
例如HRP-2中的柔性脚踝。IMU通常会提供安装位置处的角速度和线加速度,其测量仅足以测量偏航角。

最近的方法集中在融合IMU信息,力传感器数据和运动学测量来实现高保真的状态估计。IMU和运动学数据的融合最开始是用来实现里程计[1][2]。
之后,融合两者的高频率状态估计被用来估计速度并给闭环控制器使用:这可以在运动学层面进行[3][4][5],也可以扩展到动力学模型[6][7]。
其他传感器包括激光雷达的数据也可以被融合进来[8]。这些方法的共性是,解一个全耦合的推理问题来从传感器测量中估计位置和姿态。
然而这种全耦合的推理计算十分耗时,同时很难调参,并且通常代码实现和debug是很复杂的。

也有更简单的状态估计器和平衡控制器一起被提出,并且通常可以实现很震撼的实物效果[9][10][11][12][13]。这些估计器将姿态和位置估计解耦
来简化问题,这就引出了一个问题:为了实现高性能的控制效果,我们在多大程度上需要全耦合的位姿解算?现在还很难回答这个问题,因为这些方法
的严格验证有限,并且从来没有进行过合适的基准测试/对比试验(have never been properly benchmarked)。

这篇文章的主要贡献在于在WuKong4上进行了原地平衡和行走实验,并且对比(benchmark)了两种简单的状态估计方法。对比试验同样考察了不同传感器
的贡献度,比如IMU数据的加入,以及用力传感器来检测接触扰动。我们在第二和第三部分提出了两种不同的简化状态估计方法;
第四部分展示并讨论了实验结果;第五部分给出了实验结论。

\subsection{方法1.两阶段加权平均(WA)}

\subsection{方法2.位置速度线性卡尔曼滤波器}
本文使用的第二种方法利用卡尔曼滤波实现了自适应权重的调整,从而得到浮动基的速度和位置估计。现在的状态估计方法已经用扩展卡尔曼滤波(EKF)
在人形和四足机器人的浮动基状态估计上取得了很好的效果[4][5]。尽管在很多实例中EKF类的方法都表现很好,但它们的误差动力学缺乏稳定性的保证,
并且在实际中可能很难调整参数。第三部分讨论了浮动基状态估计如何能在一个轻微的假设下分解成分级的姿态和位置滤波器,这是一个关键的结论。
有了这个结论之后,浮动基的位置和速度估计就转化成了一个线性的问题,我们就能用线性卡尔曼滤波器来处理这个问题。

A 过程模型
和[3]中一样,机器人的位置和速度估计的状态量$x$将机器人足的位置也包括了进去:

\section{触地状态估计}










\chapter{双足机器人质心轨迹优化}
